\documentclass{article}
\usepackage{amsthm}
\usepackage{amssymb}
\usepackage{amsmath}
\usepackage{fancyheadings}

\newcommand{\ba}[1]{\begin{align*}    #1    \end{align*}}
\newcommand{\ban}[1]{\begin{align}    #1    \end{align}}
\renewcommand{\vec}[1]{\mathbf{#1}}\setlength{\headsep}{1in}
\renewcommand{\t}[1]{\texttt{#1}}
\pagestyle{fancy}

\lhead{ridenour@kth.se}
\rhead{Grid Generator, Prerequisites}
\title{Grid Generator, Prerequisites\\
}
\author{Jonathan Ridenour}

\begin{document}

\maketitle

\section{Problem Statement}
Given a digital elevation model (DEM) of Earth's surface, we need a domain of point coordinates in three dimensions suitable for solving partial differential equations on. The domain extends downward from the DEM surface to a user-defined desired depth.

\section{Requirements}

\subsection{Inputs}
Data to be supplied by the user are: 
\begin{itemize}
\item a digital elevation model ($m$ above mean sea level) covering the desired area e.g.\ a GTOPO30 tile, 
\item longitude and latitude intervals (in degrees easting and northing),
\item a desired depth ($m$).
\end{itemize}

\subsection{Outputs}
Data to be output by the program are three tab-delimited .txt files containing the separate $(x, y, z)$-coordinates of a computational grid spanning the longitude and latitude limits and down to the desired depth.

\section{Architecture}

\subsection{Class Descriptions}

\subsubsection{Point}
Data members are three doubles that constitute an $(x,y,z)$-coordinate. Public routines are getters, setters, and a \texttt{showCoordinate()} function, which displays the coordinate of the point in the terminal.

\subsubsection{Line}
Data members are an integer size \t{N} indicating the number of points in the line, and \t{coordinates}, a dynamically-allocated (size \t{N}) array of \t{Point} objects. Points are added to the line by means of a function \t{addPoint(int i, Point p)}, which inserts the Point \t{p} into \t{coordinates} at index \t{i} (the first element having index 0). Public routines are \t{addPoint(int i, Point p)} and \t{showCoordinates()}, which displays all the coordinates in the line in the terminal.

\subsubsection{Surface}

\subsubsection{Domain}

\subsection{Mathematics}

\subsubsection{2D Transfinite interpolation}
\ba{
x(\xi,\eta) = &(1-\xi)x(0,\eta)+\xi x(1,\eta)+(1-\eta)x(\xi,0)+\eta x(\xi,1)-(1-\eta)(1-\xi)x(0,0)
\\&-\xi(1-\eta)x(1,0)-(1-\xi)\eta x(0,1)-\eta \xi x(1,1),\\
\\y(\xi, \eta) = &(1-\xi)y(0,\eta)+\xi y(1,\eta)+(1-\eta)y(\xi,0)+\eta y(\xi,1)-(1-\eta)(1-\xi)y(0,0)
\\&-\xi(1-\eta)y(1,0)-(1-\xi)\eta y(0,1)-\eta \xi y(1,1).
}

\subsubsection{3D Transfinite Interpolation}
As in \cite{smith}, the formula for 3D transfinite interpolation is as follows:
\ba{
U(\xi, \eta, \zeta) = & \ (1-\xi)X(0,\eta, \zeta) + \xi X(1,\eta, \zeta), \\
V(\xi, \eta, \zeta) = & \ (1-\eta)X(\xi, 0, \zeta) + \eta X(\xi, 1, \zeta), \\
W(\xi, \eta, \zeta) = & \ (1-\zeta)X(\xi, \eta, 0) + \zeta X(\xi, \eta, 1), \\
UW(\xi, \eta, \zeta) = & \ (1-\xi)(1-\zeta) X(0, \eta, 0) + \zeta (1-\xi) X(0, \eta, 1)  \\
+ & \ \xi (1-\zeta) X(1,\eta, 0) + \xi \zeta X(1, \eta, 1), \\
UV(\xi, \eta, \zeta) = & \ (1-\xi)(1-\eta)X(0,0,\zeta) + \eta (1-\xi) X(0,1,\zeta)  \\
+ & \ \xi (1-\eta) X(1,0,\zeta) + \xi \eta X(1,1,\zeta), \\
VW(\xi, \eta, \zeta) = & \ (1-\eta)(1-\zeta)X(\xi,0,0) + \zeta (1-\eta) X(\xi, 0, 1)  \\
+ & \ \eta (1-\zeta) X(\xi, 1, 0) + \eta \zeta X(\xi, 1,1),  \\
UVW(\xi, \eta, \zeta) = & \ (1-\xi)(1-\eta)(1-\zeta)X(0,0,0) + (1-\xi)(1-\eta) \zeta X(0,0,1)  \\
+ & \ (1-\xi) \eta (1-\zeta) X(0,1,0) + \xi (1-\eta)(1- \zeta) X(1,0,0) \\
+ & \ (1-\xi) \eta \zeta X(0,1,1) + \xi (1-\eta) \zeta X(1,0,1) \\
+ & \ (1-\zeta) \xi \eta  X(1,1,0) + \xi \eta \zeta  X(1,1,1).
}
Putting these together gives the complete formula:
\ba{
X(\xi, \eta, \zeta) = & \  U(\xi, \eta, \zeta) + V(\xi, \eta, \zeta) + W(\xi, \eta, \zeta) \\
 - & \ UW(\xi, \eta, \zeta) - UV(\xi, \eta, \zeta) - VW(\xi, \eta, \zeta) \\
 + & \  UVW(\xi, \eta, \zeta).
}


\begin{thebibliography}{}
\bibitem{smith} Smith, Robert E (1998) Transfinite Interpolation Generation Systems. In Nigel P., et. al. {\it Handbook of Grid Generation}, CRC Press.
\end{thebibliography}

\end{document}
