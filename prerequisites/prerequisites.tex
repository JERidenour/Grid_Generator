\documentclass{article}
\usepackage{amsthm}
\usepackage{amssymb}
\usepackage{amsmath}
\usepackage{fancyheadings}

\newcommand{\ba}[1]{\begin{align*}    #1    \end{align*}}
\newcommand{\ban}[1]{\begin{align}    #1    \end{align}}
\renewcommand{\vec}[1]{\mathbf{#1}}\setlength{\headsep}{1in}
\renewcommand{\t}[1]{\texttt{#1}}
\pagestyle{fancy}

\lhead{ridenour@kth.se}
\rhead{Grid Generator, Prerequisites}
\title{Grid Generator, Prerequisites\\
}
\author{Jonathan Ridenour}

\begin{document}

\maketitle

\section{Problem Statement}
Given a digital elevation model (DEM) of Earth's surface, we need a domain of point coordinates in three dimensions suitable for solving partial differential equations on. The domain extends downward from the DEM surface to a user-defined desired depth.

\section{Requirements}

\subsection{Inputs}
Data to be supplied by the user are: 
\begin{itemize}
\item a digital elevation model ($m$ above mean sea level) covering the desired area e.g.\ a GTOPO30 tile, 
\item longitude and latitude intervals (in degrees easting and northing),
\item a desired depth ($m$).
\end{itemize}

\subsection{Outputs}
Data to be output by the program are three tab-delimited .txt files containing the separate $(x, y, z)$-coordinates of a computational grid spanning the longitude and latitude limits and down to the desired depth.

\section{Architecture}

\subsection{Class Descriptions}

\subsubsection{Domain}
The \t{Domain} class is the base class for the 1-, 2-, and 3-dimensional domains from which the \t{Line}, \t{Surface}, and \t{Grid} classes inherit their common functions and data members: numerical constants \t{Nx}, \t{Ny}, and \t{Nz} (number of array elements (\t{Point} objects) on corresponding sides of the grid block), \t{xi}, \t{eta}, and \t{zeta} (interpolation constants, see Section \ref{sec:Mathematics}), and dynamically allocated arrays of \t{Point} objects for coordinates and corner points. 

Routines include getters and setters, \t{showCoordinates()} (virtual) for printing the coordinates of a domain to the terminal, and \t{printCoordinates()} (virtual) for printing the coordinates to file. The $(x,y,z)$-coordinates of the domains are printed to three separate files called \t{filenameX.txt}, \t{filenameY.txt}, etc.

\subsubsection{Grid}
This is the 3-dimensional domain which is the goal of this program.  The \t{Grid} object defines a 6-sided computational domain in curvilinear coordinates. The \t{interpolate()} routine accomplishes 3D transfinite interpolation, as discussed in Section \ref{sec:3Dtfi}. The \t{coordinates} array is of size \t{Nx} $\times$ \t{Ny} $\times$ \t{Nz}, and the \t{corners} array is of size 8.

\subsubsection{Surface}
This is the 2-dimensional domain which constitutes a boundary for the 3-dimensional domain. The \t{interpolate()} routine accomplishes 2D transfinite interpolation, as discussed in Section \ref{sec:2Dtfi}. The \t{coordinates} array is of size \t{Nx} $\times$ \t{Ny}, and the \t{corners} array is of size 4.

The \texttt{Surface} class only interpolates on the $(x,y)$-plane. For this reason, a surface object should be initialized with a norm, which defines the base vector normal to the surface as if the surface were flat. The four lines which make up the sides of the surface are then projected onto the $(x,y)$-plane for interpolation. Before the surface points are returned they are reprojected back to the correct norm.

\subsubsection{Line}
This is the 1-dimensional domain which constitutes a boundary for the 2-dimensional domain. The \t{coordinates} array is of size \t{Nx}, and the \t{corners} array is of size 2. The idea with \t{Line} is to initialise it to a specific length, then fill \t{coordinates} with \t{Point} objects generated from the DEM. 

\subsubsection{Point}
Data members are three doubles that constitute an $(x,y,z)$-coordinate. 

Public routines are getters, setters, and a \texttt{showCoordinate()} function, which displays the coordinate of the point in the terminal.

\subsection{Mathematics}
\label{sec:Mathematics}

\subsubsection{2D Transfinite interpolation}
\label{sec:2Dtfi}

\ba{
x(\xi,\eta) = &(1-\xi)x(0,\eta)+\xi x(1,\eta)+(1-\eta)x(\xi,0)+\eta x(\xi,1)-(1-\eta)(1-\xi)x(0,0)
\\&-\xi(1-\eta)x(1,0)-(1-\xi)\eta x(0,1)-\eta \xi x(1,1),\\
\\y(\xi, \eta) = &(1-\xi)y(0,\eta)+\xi y(1,\eta)+(1-\eta)y(\xi,0)+\eta y(\xi,1)-(1-\eta)(1-\xi)y(0,0)
\\&-\xi(1-\eta)y(1,0)-(1-\xi)\eta y(0,1)-\eta \xi y(1,1).
}

\subsubsection{3D Transfinite Interpolation}
\label{sec:3Dtfi}

As in \cite{smith}, the formula for 3D transfinite interpolation is as follows:
\ba{
U(\xi, \eta, \zeta) = & \ (1-\xi)X(0,\eta, \zeta) + \xi X(1,\eta, \zeta), \\
V(\xi, \eta, \zeta) = & \ (1-\eta)X(\xi, 0, \zeta) + \eta X(\xi, 1, \zeta), \\
W(\xi, \eta, \zeta) = & \ (1-\zeta)X(\xi, \eta, 0) + \zeta X(\xi, \eta, 1), \\
UW(\xi, \eta, \zeta) = & \ (1-\xi)(1-\zeta) X(0, \eta, 0) + \zeta (1-\xi) X(0, \eta, 1)  \\
+ & \ \xi (1-\zeta) X(1,\eta, 0) + \xi \zeta X(1, \eta, 1), \\
UV(\xi, \eta, \zeta) = & \ (1-\xi)(1-\eta)X(0,0,\zeta) + \eta (1-\xi) X(0,1,\zeta)  \\
+ & \ \xi (1-\eta) X(1,0,\zeta) + \xi \eta X(1,1,\zeta), \\
VW(\xi, \eta, \zeta) = & \ (1-\eta)(1-\zeta)X(\xi,0,0) + \zeta (1-\eta) X(\xi, 0, 1)  \\
+ & \ \eta (1-\zeta) X(\xi, 1, 0) + \eta \zeta X(\xi, 1,1),  \\
UVW(\xi, \eta, \zeta) = & \ (1-\xi)(1-\eta)(1-\zeta)X(0,0,0) + (1-\xi)(1-\eta) \zeta X(0,0,1)  \\
+ & \ (1-\xi) \eta (1-\zeta) X(0,1,0) + \xi (1-\eta)(1- \zeta) X(1,0,0) \\
+ & \ (1-\xi) \eta \zeta X(0,1,1) + \xi (1-\eta) \zeta X(1,0,1) \\
+ & \ (1-\zeta) \xi \eta  X(1,1,0) + \xi \eta \zeta  X(1,1,1).
}
Putting these together gives the complete formula:
\ba{
X(\xi, \eta, \zeta) = & \  U(\xi, \eta, \zeta) + V(\xi, \eta, \zeta) + W(\xi, \eta, \zeta) \\
 - & \ UW(\xi, \eta, \zeta) - UV(\xi, \eta, \zeta) - VW(\xi, \eta, \zeta) \\
 + & \  UVW(\xi, \eta, \zeta).
}


\begin{thebibliography}{}
\bibitem{smith} Smith, Robert E (1998) Transfinite Interpolation Generation Systems. In Nigel P., et. al. {\it Handbook of Grid Generation}, CRC Press.
\end{thebibliography}

\end{document}
